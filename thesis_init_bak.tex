\documentclass{seal_thesis}

\usepackage{tikz}
\usepackage{tikz-qtree}
\usepackage{pgfgantt}
\usepackage{float}
\usepackage[inline]{enumitem}

\thesisType{Master Thesis}
\date{\today}
\title{Business Process Management}
\subtitle{Discrete Event Simulation For Optimal Role Resolution In Workflow Processes}
\author{Filip Ko\v{c}ovski}
\home{Lugano} % Geburtsort
\country{Switzerland}
\legi{10-932-994}
\prof{Prof. Dr. Daning Hu}
\assistent{Dr. Markus Uhr}
\email{filip.kocovski@uzh.ch}
\begindate{November 15, 2016}
\enddate{May 15, 2017}

\begin{document}
\maketitle

\frontmatter

\begin{acknowledgements}

\end{acknowledgements}

\begin{abstract}
	This study has two major purposes: 
	\begin{enumerate*}[font=\bfseries]
		\item to investigate the economic effects of applying business intelligence methods, such as IT work-flow management process automation, on classical business problems, such as the acquisition process of real estate properties and
		\item to demonstrate that an effective modeling and implementation of such acquisition process into a work-flow engine can help to increase efficiency and reduce costs.
	\end{enumerate*} 

	The acquisition process of real estate properties is a classical business mechanism that many companies have to manage. This kind of process can become a very cumbersome force to reckon with.
	
	From a business intelligence point of view, this very process is well suited for IT augmentation.
	
	The subject of this study was an existing acquisition process of real estate properties which was analyzed, from which further business requirements were analyzed and upon them IT requirements were inferred. These IT requirements then served as a basis for a modeling of a business work-flow management process based on BPMN. Eventually relational data was gathered and efficiency analysis was conducted.
	
	The approach followed for this thesis was a multi way path, where the problem was tackled from two different points of view:
	\begin{enumerate*}[font=\bfseries]
		\item Initially a profound literature review in the fields of business economics, acquisition processes of real estate properties, business intelligence and BPMN has been made and
		\item based on these literature fundamentals, a more practical approach was undertaken, where existing modeling of acquisition processes for real estate properties were analyzed and eventually extended.
	\end{enumerate*} 

	The results suggest that efficiency improvement, redundancy elimination, process optimization and costs savings are possible.
	
	INTERNAL USE: BASED ON \cite{Bond2009}
\end{abstract}

\begin{zusammenfassung}

\end{zusammenfassung}

\tableofcontents
\listoffigures
\listoftables
\lstlistoflistings

\pagebreak

\mainmatter

INTERNAL USE: BASED ON \cite{Search&Write2013}

\chapter{Introduction}
\section{Background}
\section{Thesis Scope Definition}
\section{Schematic Outline}
\subsection{Project Plan}
\label{subsec:project_plan}

This section serves as an overview of the different tasks for the project. What follows is a verbal definition of the different tasks and the respective project plan in means of Gantt charts.

Figure \ref{fig:first_plan} depicts the project plan ranging from October 1, 2016 until December 31, 2016. For sake of simplicity, the whole project plan has been split in two in order to improve readability.

The first part of the project plan is subdivided in three main groups
\begin{description}
	\item[Admin.] This group encompasses Task 1 and Task 2.
	\begin{description}
		\item[Task 1] The idea was discussed with immopac ag and approved. The discussion proceeded in an iterative fashion, where different topics have been brainstormed.
		\item[Task 2] The idea proposal was presented to Prof. Dr. Daning Hu, who provided invaluable feedback.
	\end{description}
	\item[M 1] November 15, 2016 marks the official start of this thesis and is thus marked as the first milestone.
	\item[Draft and Lit.] This group directly depends on milestone one and groups Task 3, Task 4 and Task 5. This group focuses on the initial literature research in the economic field and the work-flow management.
	\begin{description}
		\item[Taks 3] Literature research on work-flow process optimization and business and economics aspects.
		\item[Task 4] Thesis template, abstract definition and structure.
		\item[Task 5] Initial draft of related work section based on the previous literature research.
	\end{description}
	\item[Taks 6] This task is a standalone tasks that doesn't belong to any of the previous groups but depends on Task 5. It concerns an initial software and tool analysis for the later work-flow implementation.
\end{description}

\begin{figure}[H]
	\label{fig:first_plan}
\noindent\resizebox{\textwidth}{!}{
	\begin{ganttchart}[hgrid,time slot format=little-endian,x unit=.25cm,milestone/.append style={fill=red}]{01.10.2016}{31.12.2016}
		\gantttitlecalendar{month=name}\\
		\ganttgroup[group/.append style={draw=black, fill=green!75}]{Admin.}{01.10.2016}{15.11.2016}\\
		\ganttbar[bar/.append style={draw=black, fill=green!50}]{Task 1}{01.10.2016}{14.11.2016}\\
		\ganttbar[bar/.append style={draw=black, fill=green!50}]{Task 2}{15.10.2016}{14.11.2016}\\
		\ganttmilestone{M 1}{15.11.2016}\\
		\ganttgroup[group/.append style={draw=black, fill=orange!75}]{Draft and Lit.}{16.11.2016}{14.12.2016}\\
		\ganttbar[bar/.append style={draw=black, fill=orange!50}]{Task 3}{16.11.2016}{01.12.2016}\\
		\ganttbar[bar/.append style={draw=black, fill=orange!50}]{Task 4}{02.12.2016}{14.12.2016}\\
		\ganttbar[bar/.append style={draw=black, fill=orange!50}]{Task 5}{02.12.2016}{14.12.2016}\\
		\ganttbar[bar/.append style={draw=black, fill=black!50}]{Task 6}{15.12.2016}{31.12.2016}
		\ganttlink{elem3}{elem5}
		\ganttlink{elem7}{elem8}
		\ganttlink{elem5}{elem7}
	\end{ganttchart}	
}
\caption{Project Plan from October 2016 until December 2016}
\end{figure}

The second part has one main group

\begin{description}
	\item[Impl.] This group comprises Task 7, Task 8 and Task 9. Its core scope is the implementation and modeling of the work-flow process following an iterative and continuous implementation approach.
	\begin{description}
		\item[Task 7] In this period a business requirements analysis will be conducted in order to extrapolate the business needs and translate them into IT requirements for the later modeling phase.
		\item[Task 8] This central tasks embodies the actual work-flow modeling and strictly depends on the correct business analysis depicted in the previous task.
		\item[Task 9] Upon competition of the modeling,  data collection phase begins, where relational data is to be gathered for further analysis on possible efficiency improvement.
	\end{description}
	\item[Task 10] This task directly depends on the data collection phase and aims at analyzing and uncovering key insight.
	\item[Task 11] Last but not least, this period focuses on ultimating the project report and preparing the defense presentation.
	\item[M 2] Milestone two is the official closing date for the master thesis.
\end{description}

\begin{figure}[H]

\noindent\resizebox{\textwidth}{!}{
	\begin{ganttchart}[hgrid,time slot format=little-endian,x unit=.25cm,milestone/.append style={fill=red}]{01.01.2017}{15.05.2017}
		\gantttitlecalendar{month=name}\\
		\ganttgroup[group/.append style={draw=black, fill=blue!75}]{Impl.}{01.01.2017}{01.04.2017}\\
		\ganttbar[bar/.append style={draw=black, fill=blue!50}]{Task 7}{01.01.2017}{14.01.2017}\\
		\ganttlinkedbar[bar/.append style={draw=black, fill=blue!50}]{Task 8}{15.01.2017}{01.03.2017}\\
		\ganttlinkedbar[bar/.append style={draw=black, fill=blue!50}]{Task 9}{02.03.2017}{01.04.2017}\\
		\ganttlinkedbar[bar/.append style={draw=black, fill=black!50}]{Task 10}{02.04.2017}{15.04.2017}\\
		\ganttlinkedbar[bar/.append style={draw=black, fill=black!50}]{Task 11}{16.04.2017}{15.05.2017}\\
		\ganttlinkedmilestone{M 2}{15.05.2017}
	\end{ganttchart}	
}
	\caption{Project Plan from January 2017 until mid May 2017}
\end{figure}

\nocite{*}

\chapter{Theory}
\section{Literature Research}
\subsection{Economics and Real Estate Literature}
\subsection{Business Intelligence and BPMN Literature}

\chapter{Methods}
\section{Existing Acquisition Process for Real Estate Properties}
\section{Modeling of Acquisition Process for Real Estate Properties}
\section{Data Collection}

\chapter{Analysis and Results}
\section{Relational Data Analysis}

\chapter{Discussion}

\chapter{Conclusion}
\appendix

\backmatter

\bibliographystyle{alpha}
\bibliography{sources}

\end{document}
