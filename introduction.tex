\chapter{Introduction}
\label{ch:intro}

\glspl{wfms} empower business process automation \citep{Zeng2005}. One crucial aspect of this empowerment consists in effectively assigning jobs to users in \glspl{wfms} \citep{Zeng2005}. This concept is termed role resolution in the literature \citep{Cheng2000}. Solving the role resolution problem in \glspl{wfms} can lead to fully realize the potential gains such as:
\begin{enumerate*}
	\item Cost savings
	\item Fairness in workload assignment
	\item Optimal resources usage.
\end{enumerate*}

Currently most \glspl{wfms} implement elementary role resolution policies \citep{Zeng2005}. Expanding on \citet{Zeng2005}'s work, this thesis focuses on further developing traditional \gls{milp} based approaches and proposes novel \gls{rl} based methods for solving the role resolution problem in \glspl{wfms}.


\section{Objectives}
\label{sec:objectives}

The objectives of this thesis build upon the work of \citet{Zeng2005} in which they describe policies for optimal role resolution in \glspl{wfms} and extends these capabilities from a twofold perspective:
\begin{enumerate*}
	\item Further develops the \gls{milp} based approaches proposed by \citet{Zeng2005}
	\item Explores the capabilities offered by \gls{rl} in order to overcome the deterministic limitations of traditional \gls{milp} based methods by approaching role resolution in \glspl{wfms} from a stochastic perspective.
\end{enumerate*}
Formally, this thesis tries to answer the following research questions:

\begin{enumerate}[label=\textbf{Q. \Roman*},ref=Research Question \Roman*]
\item Do traditional \gls{milp} based methods for role resolution in \glspl{wfms} exhibit further optimization potential? \label{rq:one}
\item Do cutting edge alternative approaches for role resolution in \glspl{wfms} exist? \label{rq:two}
\end{enumerate}

\section{Thesis Structure}

This thesis is subdivided in seven main chapters:

\begin{itemize}
	\item \chpref{ch:related_work} gives an overview on the state of the art literature.
	\item \chpref{ch:foundations} outlines the theoretical foundations in \gls{bpmn}, \glspl{wfms}, their limitations and role resolution. 
	\item \chpref{ch:discrete_event_sim} maps the theoretical concepts outlined in \chpref{ch:foundations} with the actual discrete event simulation environment used for evaluation. It explains in detail the governing logic of the subset of \gls{bpmn} elements implemented and the role resolution approach used.
	\item \chpref{ch:policies} initially describes in detail the \gls{milp} based approach and how traditional methods have been extended, followed by a theoretical introduction in \gls{rl} which is eventually used in the culminating part where \gls{rl} based approaches are explained.
	\item \chpref{ch:empirical_analysis} describes the evaluation framework implemented, how both role resolution approaches are simulated and defines the \glspl{kpi} used to measure them.
	\item \chpref{ch:results} initially exposes the results obtained by the \gls{milp} based methods and then compares them with the traditional methods of \citet{Zeng2005}. Subsequently, the results of the \gls{rl} based methods are outlined.
	\item \chpref{ch:conclusion} is the culminating chapter in which the results obtained in \chpref{ch:results} are critically discussed and interpreted,	a summary of the key findings is outlined and the research questions posed in \secref{sec:objectives} are answered. Finally consequences and outlooks about future trends and how the empirical results can be extended by prospective researchers are given.
\end{itemize}