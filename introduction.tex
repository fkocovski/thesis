\chapter{Introduction}
\label{ch:intro}

\section{Problem Definition}

Workflows are IT solutions that can help increase efficiency and get tasks done better and faster \citep{Mentzas2001}. However, a key element of each workflow process still remains the human aspect which can take many facets, such as humans analyzing, designing or executing workflow processes. This thesis focuses on the latter: human agents interacting with workflow processes in order to work on tasks. A business process that has been efficiently analyzed and subsequently optimally implemented still cannot ensure optimal execution, or no optimal execution can be achieved while a human intervention for task execution is present. It is here that optimal role resolution comes in play: optimally choosing and assigning a specific task inside the workflow process to the best possible actor is a non trivial task that has to be solved in order to close the ``optimization'' circle that workflow engines advertise.

This field is relevant since an optimal role resolution can bring optimization from many sides:
\begin{enumerate*}
	\item Cost savings
	\item Fairness in workload assignment
	\item Optimal resources usage.
\end{enumerate*}

Currently different workflow engines exist, ranging from complete fully functional suites down to extensible frameworks that allow the implementer to adapt it to its own needs. Nonetheless, all these solutions lack optimality in the task assignment sector.

\section{Objectives}
\label{sec:objectives}

The objectives of this thesis build upon the work of \citet{Zeng2005} in which they depicted preliminary policies for optimal role resolution and extends these capabilities from a twofold perspective:
\begin{enumerate*}
	\item Further develops the mathematical premises and extends the capabilities of the batch policies proposed by \citet{Zeng2005}
	\item Explores the capabilities offered by \gls{rl} as addition and improvement for even preciser, faster and better task assignment.
\end{enumerate*}

Formally, this thesis tries to answer the following research questions:

\begin{enumerate}[label=\textbf{Q. \Roman*},ref=Research Question \Roman*]
\item Can current optimization methods for job assignment in workflow processes be further developed? \label{rq:one}
\item Are there state of the art approaches that can complement job assignment with mathematical optimization methods? \label{rq:two}
\end{enumerate}

\section{Thesis Structure}

This thesis is subdivided in five main chapters:

\begin{itemize}
	\item \chpref{ch:intro} gives an overview of why the chosen topic is relevant, what is the current context of the work and how this work fits in. It moreover articulates the central research questions that permeate this thesis and gives an overview of this essay.
	\item \chpref{ch:related_work} gives an overview of the most important conceptual definitions and the state of the art literature review in the touched thematic topics of this work.
	\item \chpref{ch:discrete_event_sim} gives an overview of the approach used for the research \eg the analysis environment and the used tools, states the hypothesis that wants to be proved and eventually describes statistically and qualitatively the data sets upon which the methodology is applied.
	\item \chpref{ch:empirical_analysis} builds upon \chpref{ch:discrete_event_sim} and makes its way into the hypothesis test field and the respective analysis results. Furthermore looks introspectively on the data correlation and gives an interpretation of the latter. Eventually in this section a statement about the contribution that the results bring into this field is given.
	\item \chpref{ch:conclusion} is the culminating chapter in which a summary of the key findings of the thesis are outlined, the research questions posed in \secref{sec:objectives} are answered by looking at the actual usability, limitations and to whom the results are most applicable. Finally outlooks about the future trends and how the empirical results of this thesis can be extended by prospective researchers is given.
\end{itemize}

\section{Project Management}

The first half of the thesis road-map consisted of an in-depth literature review with a twofold focus:
\begin{enumerate*}
	\item Analyzing existing literature and solutions in the domain as proposed by \citet{Zeng2005}.
	\item Analyzing novel approaches for complementing the existing solutions as outlined by \citet{Sutton2017}.
\end{enumerate*}

With a solid foundations laid down, starting December 1, 2016 the implementation phase began. Initially the focus was set on implementing the simulation environment, followed by the implementation of the optimization solution \ie the \gls{msa} formulation, proposed by \citet{Zeng2005} and subsequently the optimization alternatives (refer to \subsecref{sec:opt_policies}).

\fig[\textwidth]{project_management_gantt_chart}{Project Management Gantt Chart for 2016}{fig:project_management_gantt_chart}

Starting January 1, 2017 the implementation focus shifted towards the novel \gls{rl} approaches as outlined by \citet{Sutton2017}, which continued until March 31, 2017. In parallel, the thesis report was being iteratively written. Starting April 17, 2017 the presentation's work started in conjunction with the thesis report.

\fig[\textwidth]{project_management_gantt_chart_2017}{Project Management Gantt Chart for 2017}{fig:project_management_gantt_chart_2017}