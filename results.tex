\chapter{Results}
\label{ch:results}

This chapter outlines the results obtained by the \gls{milp} approach in \secref{sec:opt_results} respectively the \gls{rl} approach in \secref{sec:rl_results} following the guidelines defined in \chpref{ch:empirical_analysis}.

All simulations have been tested with different combinations of global variables \ie number of users, service interval, generation interval, length of simulation time, batch size (where it applies), task variability, worker variability and random state seed (where it applies). For ease of reading purposes, the global variables have been set to the following parameters according to the default column in \tabref{tab:global_parameters_sim}. The rightmost column in \tabref{tab:global_parameters_sim} indicates the validity range for each variable.

% Please add the following required packages to your document preamble:
% \usepackage{booktabs}
\begin{table}[!ht]
	\centering
		\begin{tabular}{@{}lll@{}}
		\toprule
		Variable            & Default     & Valid Range \\ \midrule
		Number of Users     & $5$           & $1-\infty$      \\
		Service Interval    & $1$           & $1-\infty$      \\
		Generation Interval & $3$           & $1-\infty$      \\
		Simulation Time     & $1000$          & $1-\infty$      \\
		Batch Size          & $5$ ($1$ for 1-Batch-1)           & $1-\infty$      \\
		Task Varaibility    & $20\%$ of service interval       & $0\%-100\%$      \\
		Worker Variability  & $20\%$ of service interval       & $0\%-100\%$      \\
		Random State Seed   & $2$           & $\emptyset-\infty$      \\
		Process    & Acquisition & Acquisition, Simple      \\ \bottomrule
		\end{tabular}
	\caption{Global parameters for simulation}
	\label{tab:global_parameters_sim}
\end{table}

\section{\glsentrylong{milp}}
\label{sec:opt_results}

\citet[pp. 18--22]{Zeng2005} outline how different global parameters configurations and policy usage can affect \glspl{kpi}. They summarize their key findings as follows:
\begin{enumerate*}
	\item Usage of batch optimization should be done only with medium to high system load \citep[p. 24]{Zeng2005}
	\item Batch optimization policies without a fixed batch size, such as 1-Batch-1 yield best results \citep[p. 24]{Zeng2005}.
\end{enumerate*}

In order to assert the validity of the interpretation of \citet{Zeng2005}'s works and all subsequent derivative policies a comparison with similar configurations has been made for all five \gls{milp} based policies. \citet{Zeng2005}'s main efficiency parameter is defined as the maximum flowtime or in their own words: ``In business terms, maximum flowtime represents the guaranteed response time across tasks, indicating the quality of services'' \citep[p. 17]{Zeng2005}. In this study, the comparable parameter used to evaluate a policy's efficiency is called lateness and has been previously defined in \equref{eq:lateness}. In regards to lateness, \figref{fig:opt_kpis_comp} shows that akin results to \citet{Zeng2005} are obtained.

\fig[\textwidth]{opt_kpis_comp}{\glsentryshortpl{kpi}  comparison for different \gls{milp} based policies using \glsentryshort{msa}}{fig:opt_kpis_comp}  

The simulations have been run with the parameters outlined in \tabref{tab:global_parameters_sim} by using the same formulation used by \citet{Zeng2005}: \gls{msa}.

By running the same simulations with the \gls{milp} based formulation implemented for this thesis \ie \gls{st}, ceteris paribus, the summarized \glspl{kpi} among all \gls{milp} based policies can be seen in \figref{fig:opt_st_kpis_comp}.

\fig[\textwidth]{opt_st_kpis_comp}{\glsentryshortpl{kpi}  comparison for different \gls{milp} based policies using \glsentryshort{st}}{fig:opt_st_kpis_comp}

The speedup for both batch policies with a higher batch size is tiny, but when considering the speedup between the 1-Batch-1 policy with \gls{msa} and \gls{st}, a wealthy speedup is present for all \glspl{kpi}. For a detailed overview of the overall speedups of \gls{st} against \gls{msa} refer to \figref{fig:opt_kpis_comp_gain}.

\fig[\textwidth]{opt_kpis_comp_gain}{\glsentryshortpl{kpi}  speedup comparison between \glsentryshort{msa} and \glsentryshort{st} for 5-Batch, 5-Batch-1 and 1-Batch-1}{fig:opt_kpis_comp_gain}

Astonishing speedups have been observed for the 1-Batch-1 policy\footnote{For a detailed comparison of how different batch sizes affect the policy's \glspl{kpi} refer to \subsecref{subsec:kbatch_bscomp_app} and \subsecref{subsec:kbatchone_bscomp_app}.}, which is indeed the most efficient policy as mentioned by \citet[p. 24]{Zeng2005}. \tabref{tab:opt_kpis_comp_gain} summarizes these values.

% Please add the following required packages to your document preamble:
% \usepackage{booktabs}
\begin{table}[!ht]
	\centering
		\begin{tabular}{@{}ll@{}}
		\toprule
		\glsentryshort{kpi}                 & Speedup \\ \midrule
		Lateness            & $1.23$             \\
		Wait Time           & $4.0$             \\
		Service Time        & $1.1$             \\
		Average System Load & $1.1$             \\ \bottomrule
		\end{tabular}
	\caption{Speedup across all \glsentryshortpl{kpi} for \glsentryshort{st} vs \glsentryshort{msa}}
	\label{tab:opt_kpis_comp_gain}
\end{table}

\section{\glsentrylong{rl}}
\label{sec:rl_results}

This section focuses on the results obtained with the \gls{rl} methods outlined in \secref{sec:rl_policies}. A more in-depth review of the different policies is required, thus a finer subdivision has been made in different subsections per policy type: \subsecref{subsec:rl_batch} focuses on batch policies, \subsecref{subsec:rl_llqp} focuses on \gls{llqp} policies and \subsecref{subsec:rl_others} focuses on all remaining policies that do not fit in either of the previous categories.

In order to maintain fairness amongst \gls{rl} training methods, all required parameters are globally set and equal across all simulation scripts and can be found summarized in \tabref{tab:global_rl_params} which complement the global simulation parameters depicted in \tabref{tab:global_parameters_sim}.

% Please add the following required packages to your document preamble:
% \usepackage{booktabs}
\begin{table}[!ht]
	\centering
		\begin{tabular}{@{}ll@{}}
		\toprule
		Parameter        & Value  \\ \midrule
		Discount factor $\gamma$            & $0.5$    \\
		Step size parameter $\alpha$            & $0.0001$ \\
		\glsentrylong{ep} threshold $\epsilon$          & $0.1$    \\
		\glsentryshort{mc} epochs        & $1000$   \\
		\glsentryshortpl{ann} \glsentryshort{mc} epochs & $10000$ \\
		\glsentryshort{mc} and \glsentryshortpl{ann} \glsentryshort{mc} epoch training time & $100$ \\
		\glsentryshort{td} training time & $100000$   \\ \bottomrule
		\end{tabular}
	\caption{Global \glsentryshort{rl} parameters}
	\label{tab:global_rl_params}
\end{table}

Comparisons will be made, where not otherwise stated, with the corresponding \gls{milp} based policy under the same conditions.

\subsection{Batch}
\label{subsec:rl_batch}

Five different batch policies with \gls{rl} have been developed. \tabref{tab:rl_batch_policies_overview} gives an overview.

% Please add the following required packages to your document preamble:
% \usepackage{booktabs}
% \usepackage{graphicx}
\begin{table}[!ht]
	\centering
	\resizebox{\textwidth}{!}{%
		\begin{tabular}{@{}lllll@{}}
		\toprule
		Technical Name          & Policy Type & Update Method & $Q$ Value Method               & Other Characteristics     \\ \midrule
		k\_batch\_mc\_vfa       & 1-Batch     & \glsentryshort{mc}            & \glsentryshort{vfa} & None                      \\
		k\_batch\_mc\_vfa\_op   & 1-Batch     & \glsentryshort{mc}            & \glsentryshort{vfa} & \glsentryshort{op}                \\
		k\_batch\_mc\_vfa\_opep & 1-Batch     & \glsentryshort{mc}            & \glsentryshort{vfa} & \glsentrylong{ep}, \glsentryshort{op} \\
		k\_batch\_td\_vfa\_op   & 1-Batch     & \glsentryshort{td}            & \glsentryshort{vfa} & \glsentryshort{op}                \\
		k\_batchone\_td\_vfa\_op   & 1-Batch-1     & \glsentryshort{td}            & \glsentryshort{vfa} & \glsentryshort{op}                \\ \bottomrule
		\end{tabular}%
	}
	\caption{Overview of developed batch policies with \glsentryshort{rl}}
	\label{tab:rl_batch_policies_overview}
\end{table}

\tabref{tab:rl_batch_kpis_comp_gain} summarizes the results. For the detailed results refer to \subsecref{subsec:onebatch_rl_msa_comp_app} for 1-Batch respectively to \subsecref{subsec:onebatchone_rl_msa_comp_app} for 1-Batch-1.

% Please add the following required packages to your document preamble:
% \usepackage{booktabs}
% \usepackage{graphicx}
\begin{table}[!ht]
	\centering
	\resizebox{\textwidth}{!}{%
		\begin{tabular}{@{}llllll@{}}
		\toprule
		\glsentryshort{kpi}                 & k\_batch\_mc\_vfa & k\_batch\_mc\_vfa\_op & k\_batch\_mc\_vfa\_opep & k\_batch\_td\_vfa\_op & k\_batchone\_td\_vfa\_op \\ \midrule
		Lateness            & $1.22$                  & $1.24$                      & $1.22$                        & $1.23$                      & $1.23$                         \\
		Wait Time           & $2.63$                  & $3.08$                      & $2.39$                        & $3.62$                      & $3.51$                         \\
		Service Time        & $1.11$                  & $1.11$                      & $1.12$                        & $1.1$                      & $1.1$                         \\
		Average System Load & $1.11$                  & $1.11$                      & $1.12$                        & $1.1$                      & $1.1$                         \\ \bottomrule
		\end{tabular}%
	}
	\caption{Speedup across all \glsentryshortpl{kpi} for batch policies with \glsentryshort{rl} vs \glsentryshort{msa}}
	\label{tab:rl_batch_kpis_comp_gain}
\end{table}

\subsection{\glsentrylong{llqp}}
\label{subsec:rl_llqp}

Three different \gls{llqp} policies with \gls{rl} have been developed. \tabref{tab:rl_llqp_policies_overview} gives an overview. Other policies have been implemented for evaluating different \gls{rl} methods, however they are not considered for the final evaluation. These policies can be seen in \tabref{tab:rl_llqp_add_policies_overview}.

% Please add the following required packages to your document preamble:
% \usepackage{booktabs}
% \usepackage{graphicx}
\begin{table}[!ht]
	\centering
	\resizebox{\textwidth}{!}{%
		\begin{tabular}{@{}lllll@{}}
		\toprule
		Technical Name          & Policy Type & Update Method & $Q$ Value Method               & Other Characteristics     \\ \midrule
		llqp\_mc\_vfa\_op       & \glsentryshort{llqp}     & \glsentryshort{mc}            & \glsentryshort{vfa} & \glsentryshort{op}                      \\
		llqp\_td\_vfa\_op   & \glsentryshort{llqp}     & \glsentryshort{td}            & \glsentryshort{vfa} & \glsentryshort{op}                \\
		llqp\_td\_tf\_op   & \glsentryshort{llqp}     & \glsentryshort{td}            & \glsentryshortpl{ann} & \glsentryshort{op}, \glsentryshort{1l}                \\ \bottomrule
		\end{tabular}%
	}
	\caption{Overview of developed \glsentryshort{llqp} policies with \glsentryshort{rl}}
	\label{tab:rl_llqp_policies_overview}
\end{table}

\tabref{tab:rl_llqp_kpis_comp_gain} shows the summarized results. For the detailed results refer to \subsecref{subsec:llqp_rl_msa_comp_app}.

% Please add the following required packages to your document preamble:
% \usepackage{booktabs}
\begin{table}[!ht]
	\centering
		\begin{tabular}{@{}llll@{}}
		\toprule
		\glsentryshort{kpi}                 & llqp\_mc\_vfa\_op & llqp\_td\_vfa\_op & llqp\_td\_tf\_op \\ \midrule
		Lateness            & $1.0$                  & $1.0$                  & $1.0$                 \\
		Wait Time           & $0.92$                  & $0.99$                  & $1.02$                 \\
		Service Time        & $1.01$                  & $1.0$                  & $1.0$                 \\
		Average System Load & $1.01$                  & $1.0$                  & $1.0$                 \\ \bottomrule
		\end{tabular}
	\caption{Speedup across all \glsentryshortpl{kpi} of \glsentryshort{llqp} policies with \glsentryshort{rl} vs \glsentryshort{msa}}
	\label{tab:rl_llqp_kpis_comp_gain}
\end{table}

\tabref{tab:llqp_mc_vfa_op_vs_on} shows the comparison between \gls{op} and \gls{onp} approaches.

% Please add the following required packages to your document preamble:
% \usepackage{booktabs}
\begin{table}[!ht]
	\centering
		\begin{tabular}{@{}llll@{}}
		\toprule
		\glsentryshort{kpi}                 & llqp\_mc\_vfa\_op & llqp\_mc\_vfa & Speedup \\ \midrule
		Lateness            & $1.2756$                  & $1.2914$              & $0.99$                   \\
		Wait Time           & $0.0796$                  & $0.0711$              & $1.12$                   \\
		Service Time        & $1.1960$                  & $1.2203$              & $0.98$                   \\
		Average System Load & $38.22\%$                  & $39.00\%$              & $0.98$                   \\ \bottomrule
		\end{tabular}
	\caption{\glsentryshortpl{kpi} comparison between \glsentryshort{op} and \glsentryshort{onp} approaches}
	\label{tab:llqp_mc_vfa_op_vs_on}
\end{table}

\subsection{Others}
\label{subsec:rl_others}

Three different additional policies with \gls{rl} have been developed which have been used to fully emulate the behavior of K-Batch and 1-Batch-1 (as explained in \subsecref{subsec:batch_size_emulation}). \tabref{tab:rl_others_policies_overview} gives an overview.

% Please add the following required packages to your document preamble:
% \usepackage{booktabs}
% \usepackage{graphicx}
\begin{table}[!ht]
	\centering
	\resizebox{\textwidth}{!}{%
		\begin{tabular}{@{}lllll@{}}
		\toprule
		Technical Name          & Policy Type & Update Method & $Q$ Value Method               & Other Characteristics     \\ \midrule
		wz\_td\_vfa\_op       & \glsentryshort{wz}     & \glsentryshort{td}            & \glsentryshort{vfa} & \glsentryshort{op}                      \\
		wz\_one\_td\_vfa\_op   & \glsentryshort{wzo}     & \glsentryshort{td}            & \glsentryshort{vfa} & \glsentryshort{op}                \\
		bi\_one\_mc\_tf   & \glsentryshort{bi}     & \glsentryshort{mc}            & \glsentryshortpl{ann} & \glsentryshort{pg}                \\ \bottomrule
		\end{tabular}%
	}
	\caption{Overview of additional developed policies with \glsentryshort{rl}}
	\label{tab:rl_others_policies_overview}
\end{table}

\tabref{tab:rl_others_kpis_comp_gain} shows the summarized results. For the detailed results refer to \subsecref{subsec:others_rl_msa_comp_app}.

% Please add the following required packages to your document preamble:
% \usepackage{booktabs}
% \usepackage{graphicx}
\begin{table}[!ht]
	\centering
	\resizebox{\textwidth}{!}{%
		\begin{tabular}{@{}lllllll@{}}
		\toprule
		\glsentryshort{kpi}                 & wz\_td\_vfa\_op & wz\_one\_td\_vfa\_op & bi\_one\_mc\_tf\_1l & bi\_one\_mc\_tf\_2l & bi\_one\_mc\_tf\_3l & bi\_one\_mc\_tf\_4l \\ \midrule
		Lateness            & $0.97$                & $1.2$                     & $1.07$                    & $0.9$                    & $0.84$                    & $0.84$                    \\
		Wait Time           & $0.89$                & $2.33$                     & $2.46$                    & $1.11$                    & $0.85$                    & $0.86$                    \\
		Service Time        & $1.21$                & $1.1$                     & $0.98$                    & $0.87$                    & $0.84$                    & $0.84$                    \\
		Average System Load & $1.22$                & $1.1$                     & $0.98$                    & $0.87$                    & $0.84$                    & $0.84$                    \\ \bottomrule
		\end{tabular}%
	}
	\caption{Speedup across all \glsentryshortpl{kpi} of the additional policies with \glsentryshort{rl} against the \glsentryshort{msa} formulation}
	\label{tab:rl_others_kpis_comp_gain}
\end{table}