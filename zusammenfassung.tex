\begin{otherlanguage}{ngerman}
	\begin{zusammenfassung}
	Die Zuteilung von Ressourcen in Workflowprozessen zwecks Effizienz Maximierung, insbesondere menschliche Ressourcen \dhg Nutzer, ist ein zentraler Aspekt bei der Umsetzung von Workflowinstallationen in Gesellschaftsumgebungen \citep{Mentzas2001}. Wie diese Zuteilung erfolgt oder im Allgemeinen die Lösbarkeit des Zuordnungsproblems wurden ausführlich erforscht \citep{Zeng2005}. Übliche Methoden aus der kombinatorischen Optimierung mit numerischen Optimierungen haben zu vielversprechenden Resultate geführt \citep{Zeng2005}. Allerdings spielt die rechnerische Komplexität aus der mathematischen Perspektive eine zentrale Rolle \citep{Zeng2005}.

	Diese Masterarbeit erweitert die prominente Arbeit im Bereich von der optimalen Aufgabezuteilung in Workflowprozessen indem die mathematischen Optimierungsmethoden ausgebaut wurden und einen neuartigen Ansatz mit \gls{rl} vorgestellt wurde \citep{Sutton2017}.

	Mittels einer diskreten Ereignisssimulationsumgebung worin die bestehenden Policen für die optimalen Aufgabenzuteilung in Workflowprozessen \zbg \gls{sq}, \gls{llqp}, K-Batch und 1-Batch-1, unter Verwendung von mathematischen und \gls{rl} basierten Methoden erweitert und getestet wurden. Die diskrete Ereignisssimulationsumgebung aus dieser Masterarbeit ermöglicht die Steuerung von wesentlichen Variablen die die Effizienz einer Police beeinflussen können \zbg Anzahl Nutzer, Generationsintervall, Bearbeitungsintervall und Länge der Simulation.

	Sowohl die mathematischen als auch die \gls{rl} basierten Optimierungsmethoden übertreffen die bestehenden Methoden unter verschiedenen Konditionen um einen Beschleunigungsfaktor von $1.3$. Obwohl vielversprechende Resultate erreicht wurden, muss man die Resultateinterpretation mit Vorsicht geniessen. Die mathematischen Optimierungsmethoden weisen einerseits höhere rechnerische Komplexität auf, welche Businesskompromisse aufwirft und anderseits überwinden die \gls{rl} basierten Methoden die rechnerische Komplexität von den Optimierungsmethoden. Diese verlangen jedoch langwierige Trainingssessionen um die optimale Konvergenz sicherzustellen.

	Die \gls{rl} basierten Methoden legen die Grundlagen für Erweiterungen mittels alternativer Methoden \zbg \gls{irl} \citep{Ng2000} und \gls{al} \citep{Abbeel2004}. \gls{al} verspricht die Einschränkungen von \gls{rl} basierten Methoden durch die Auflösung von internen Entlohnungsfunktionen zu überwinden und lediglich optimale Verhalten durch die ``Beobachtung'' von Sachverständigen zu lernen \citep{Abbeel2004}.
	\end{zusammenfassung}
\end{otherlanguage}